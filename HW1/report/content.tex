\section*{6.1}

I take the phrase ``Assume\dots and equal prior probabilities'' to mean that
\(P(+) = P(-)\).

Thus for each class we just multiply the word probabilities to get the
probability of that class.
\begin{equation}
  \begin{aligned}
    P(+) = 0.09 \times 0.07 \times 0.29 \times 0.04 \times 0.08 =
    \num{5.8464e-6} \\
    P(-) = 0.16 \times 0.06 \times 0.06 \times 0.15 \times 0.11 = \num{9.5040e-6}
  \end{aligned}
\end{equation}

As we can see, the Naive Bayes predicts the class to be negative.


\section*{6.2}

\hl{NOTE:\@ It seems that I used binary naive bayes in this exercise already, I
hope that's okay}

\vspace{1cm}

We'll denote the ``comedy'' class probability with \(P(C)\) and the ``action''
with \(P(A)\).

The priors are
\begin{equation}
  \begin{aligned}
    P(C) = \frac{2}{5} \\
    P(A) = \frac{3}{5}
  \end{aligned}
\end{equation}

The total number of unique words that appear in all the documents is
\( 3 + 3 + 1 + 0 + 0 = 7 \).
The number of unique words in comedy is \( 3 + 2 = 5 \).
In action \( 3 + 1 + 2 = 6 \)

The equation for determining the class probability of a word is
\begin{equation}
  P(\text{word}|\text{class}) = \frac{\text{occurrences} + 1}{\text{unique
  occurrences in the class} + \text{sum of all unique occurrences}}
\end{equation}
This probability is then multiplied by the prior to get the final probability.

The test document is
\begin{verbatim}
D = fast, couple, shoot, fly
\end{verbatim}

For comedy
\begin{equation}
  \begin{split}
    P(\text{fast}|C) = \frac{1+1}{5+7} = \frac{2}{12} \\
    P(\text{couple}|C) = \frac{1+1}{5+7} = \frac{2}{12} \\
    P(\text{shoot}|C) = \frac{0+1}{5+7} = \frac{1}{12} \\
    P(\text{fly}|C) = \frac{1+1}{5+7} = \frac{2}{12}
  \end{split}
\end{equation}

For action
\begin{equation}
  \begin{split}
    P(\text{fast}|A) = \frac{1+1}{6+7} = \frac{2}{13} \\
    P(\text{couple}|A) = \frac{0+1}{6+7} = \frac{1}{13} \\
    P(\text{shoot}|A) = \frac{1+1}{6+7} = \frac{2}{13} \\
    P(\text{fly}|A) = \frac{1+1}{6+7} = \frac{2}{13}
  \end{split}
\end{equation}

Final probabilities
\begin{equation}
  \begin{split}
    P(C)P(D|C) = \frac{2}{5} \times \frac{2^3 \times 1}{12^4} = \frac{1}{6480}
    \approx \num{1.5432e-4}\\
    P(A)P(D|A) = \frac{3}{5} \times \frac{2^3 \times 1}{13^4} =
    \frac{24}{142805} \approx \num{1.6806e-4}
  \end{split}
\end{equation}

So the predicted genre for the test document is action.
         
         
\section*{6.3}
Priors are the same with both methods.
\begin{equation}
  \begin{split}
    P(+) = \frac{2}{5} \\
    P(-) = \frac{3}{5}
  \end{split}
\end{equation}

The test sentence:
\begin{verbatim}
A good, good plot and great characters, but poor acting
\end{verbatim}

Because our training set only contains the words ``good'', ``poor'' and
``great'', the test sentence is modified to the form:
\begin{verbatim}
good good great poor
\end{verbatim}

\subsection*{Naive Bayes}
Total word occurrences: 23 \\
Positive word occurrences: 9 \\
Negative word occurrences: 14

For the positive class
\begin{equation}
  \begin{split}
    P(\text{good}|+) = \frac{3+1}{9+23} = \frac{4}{32} \\
    P(\text{great}|+) = \frac{5+1}{9+23} = \frac{6}{32} \\
    P(\text{poor}|+) = \frac{1+1}{9+23} = \frac{2}{32}
  \end{split}
\end{equation}

For the negative class
\begin{equation}
  \begin{split}
    P(\text{good}|-) = \frac{2+1}{14+23} = \frac{3}{37} \\
    P(\text{great}|-) = \frac{2+1}{14+23} = \frac{3}{37} \\
    P(\text{poor}|-) = \frac{10+1}{14+23} = \frac{11}{37}
  \end{split}
\end{equation}

Final probabilities (noting that ``good'' occurs twice)
\begin{equation}
  \begin{split}
    P(+)P(S|+) = \frac{2}{5} \times \frac{4 \times 4 \times 6 \times 2}{32^4} =
    \frac{3}{40960} \approx \num{7.3242e-5} \\
    P(-)P(S|-) = \frac{3}{5} \times \frac{3 \times 3 \times 3 \times 11}{37^4} 
    \approx \num{9.5083e-5}
  \end{split}
\end{equation}

Which means that the naive bayes recognizes this as a negative review.

\subsection*{Binarized Naive Bayes}
Binarized NB is almost the same, both we clip word counts in a document at 1.

Total word occurrences: 11 \\
Positive word occurrences: 4 \\
Negative word occurrences: 7

For the positive class
\begin{equation}
  \begin{split}
    P(\text{good}|+) = \frac{1+1}{4+11} = \frac{2}{15} \\
    P(\text{great}|+) = \frac{2+1}{4+11} = \frac{3}{15} \\
    P(\text{poor}|+) = \frac{1+1}{4+11} = \frac{2}{15}
  \end{split}
\end{equation}

For the negative class
\begin{equation}
  \begin{split}
    P(\text{good}|-) = \frac{2+1}{7+11} = \frac{3}{18} \\
    P(\text{great}|-) = \frac{1+1}{7+11} = \frac{2}{18} \\
    P(\text{poor}|-) = \frac{3+1}{7+11} = \frac{4}{18}
  \end{split}
\end{equation}

Final probabilities (noting that ``good'' occurs twice)
\begin{equation}
  \begin{split}
    P(+)P(S|+) = \frac{2}{5} \times \frac{2 \times 2 \times 3 \times 2}{15^4} =
    \frac{16}{84375} \approx \num{1.8963e-4} \\
    P(-)P(S|-) = \frac{3}{5} \times \frac{3 \times 3 \times 2 \times 4}{18^4} =
    \frac{1}{2430} \approx \num{4.1152e-4}
  \end{split}
\end{equation}

So, even the Binarized NB recognizes this as a negative review. I would guess
this to be because of the prior bias to negative reviews, and the fact that
good occurs twice in the negative reviews, and twice in the test review.


































