For this summary I read the paper called ``The effect of wording on message
propagation: Topic- and author-controlled natural experiments on Twitter'' by
C. Tan et al~\cite{DBLP:journals/corr/TanLP14}.

\section{Content Summary}

The \textbf{motivation} behind the paper is to find out how the wording of an
online message affects its success. Clearly, knowing how to craft more
successful messages is a valuable skill to a variety of entities.
The paper \textbf{contributes} multiple different ``features'' that affect the
success of a message; Asking people to share the message helps, for example.
There has been prior \textbf{related work} that has focused on how the wording
of descriptions or texts affect their success, but no large-scale studies on
the topic exist.
\textbf{Support} for their hypothesis is given by examining the Twitter data;
From there it is easy to see which features improve a message's success.


\section{Analysis}

I think the \textbf{writing} in this paper is exceptional. Contrary to most
scientific papers, it is easy to read even for someone not fully versed in the
subject, while still being informative.
The \textbf{motivation} for the paper is also valid in my opinion; Knowing how
the wording of messages helps or hinders their propagation is in my mind
clearly valuable. According to this paper at least, there hasn't been too much
research into the topic either.
The team's \textbf{contributions} seem interesting, although not terribly
surprising. However, because there hasn't been a similar study on a large
scale, even non-surprising results are important.
\textbf{Evaluating} their contributions, their dataset seems to be good enough,
and the features they pick out as important seem so according to the data and
my intuition as well.
The author's have made a version of their message classifier available for
testing. They also mention possible \textbf{future work} in their paper:
Generalizing of the mentioned features, and the psychological and cultural
mechanisms that make these features important.
